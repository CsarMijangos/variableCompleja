\section{Recordatorio de Análisis de Fourier}
El análisis de Fourier y el análisis complejo explotan las propiedades de la función 
exponencial que hemos estudiado. Estos están intimamente ligados gracias a lo siguiente:
Consideremos una serie de potencias de radio de convergencia 1, \(f(z) = \sum_{n=0}^{\infty} a_nz^n\).
Si \(\abs{z} = 1\), podemos escribir \(z = e^{it}\) con \(t\in \R\), y formalmente nuestra serie de 
potencias deviene en la siguiente serie trigonométrica \(\sum_{n=0}^{\infty}a_ne^{int}\). Este tipo 
series son delicadas de estudiar, nosotros nos contentaremos con presentar algunos resultados simples
pero robustos. Denotaremos por \(\mathcal{C}\) al espacio vectorial de funciones continuas y 
\(2\pi\)-periódicas \(f: \R \to \C \). Dotaremos a \(\mathcal{C}\) con la norma infinito definida por
\(\norm{f}_{\infty} \derDefi \sup_{0\leq t \leq 2\pi} \abs{f(t)} \). Para \(f \in \mathcal{C}\) y 
\(n\in \Z\), el \(n\)-ésimo coeficiente de Fourier, denotado por \(c_n(f)\) o \(\hat{f}(n)\) se define
por 
\[
c_n \derDefi \frac{1}{2\pi} \int_{0}^{2\pi} f(t) e^{-int} dt.
\]
Diremos que una serie trigonométrica \(\sum_{n\in \Z} a_ne^{int}\) converge uniformemente en \(\R\) si
la siguiente sucesión de sumas parciales simétricas
\[
S_N(t) = \sum_{n=-N}^{N} a_ne^{int}
\]
converge uniformemente en \(\R\) cuando \(N \to \infty\). Estamos en condiciones de establecer el siguiente
teorema.

\begin{theo}\label{teo-series-Fourier}
    Sea \(\sum_{n=-\infty}^{\infty}a_ne^{int}\) una serie trigonométrica uniformemente convergente en \(\R\),
    de suma \(F\). Entonces, \(F \in \mathcal{C}\) y, además,
    \[
    c_n(F) = a_n, \quad \forall n\in \Z.
    \]
\end{theo}
\begin{dem}
    El argumento de la demostración descansa en la ortonormalidad de las exponenciales \(e^{ijt}\) cuando 
    \(j\) recorre \(\Z\):
    \[
    \frac{1}{2\pi} \int_{0}^{2\pi} e^{ijt}e^{-ikt} dt = 
        \begin{cases}
        0, \quad \text{si} \quad j\neq k, \\
        1, \quad \text{si} \quad j=k.        
        \end{cases}
    \]
    Fijemos un \(n\in \Z\) arbitrario. Para \(N \geq \abs{n}\), utilizando la ortogonalidad de \(e^{ijt}\) tenemos que:
    \[
    c_n(S_N) = \frac{1}{2\pi} \int_{0}^{2\pi} \sum_{j=-N}^{N} a_je^{ijt} e^{-int} dt = 
               \sum_{j=-N}^{N} a_j \frac{1}{2\pi} \int_{0}^{2\pi} e^{ijt} e^{-int} dt = a_n.
    \]
    Haciendo tender \(N\to \infty\) y utilizando la convergencia uniforme se concluye la veracidad del teorema.
\end{dem}
Enunciaremos ahora una versión simple de la identidad de Parseval, que nos será suficiente en esta obra.

\begin{theo}[Parseval]\label{teo-identidad-Parseval}
    Sea \(\sum_{n=-\infty}^{\infty}a_ne^{int}\) una serie trigonométrica uniformemente convergente en \(\R\), con suma
    \(S(t)\). Entonces,
    \[
    \sum_{n=-\infty}^{\infty} \abs{a_n}^2 < \infty
    \]
    y, además,
    \[
    \sum_{n=-\infty}^{\infty} \abs{a_n}^2 = \frac{1}{2\pi} \int_{0}^{2\pi} \abs{S(t)}^2 dt.
    \]
    Si \(f(z) = \sum_{n=0}^{\infty}a_nz^{n}\) es una serie de potencias de radio de convergencia \(R\), tenemos que 
    para todo \(0\leq r < R\):
    \[
    \sum_{n=0}^{\infty}\abs{a_n}^2 r^{2n} = \frac{1}{2\pi} \int_{0}^{2\pi} \abs{f(re^{it})}^2 dt.
    \]
\end{theo}
\begin{dem}
    Utilizaremos la ortonormalidad de las exponenciales \(e^{ijt}\) cuando \(j\) recorre \(\Z\). Sea \(S_N(t) = \sum_{n=-N}^{N}a_n e^{int}\).
    Ya que tenemos que \(\abs{S_N(t)} \leq \sum_{n=-N}^{N}\abs{a_n}\), y que existe \(N_0\) tal que \(\abs{S_N(t)}\leq \abs{S(t)} + 1\) para 
    \(N\geq N_0\), y que la función continua y \(2\pi\)-periódica \(S\) es acotada en \(\R\), se sigue que \(\abs{S_N(t)} \leq M\), para algún 
    \(M\) que es independiente de \(N\) y \(t\). Además, por ortogonalidad antes mencionada tenemos:
    \[
    M^2 \geq \frac{1}{2\pi}\int_{0}^{2\pi} \abs{S_N(t)}^2 dt = \frac{1}{2\pi} \int_{0}^{2\pi}\sum_{n=-N}^{N} a_j \conj{a_k} e^{ijt}e^{-ikt} dt 
    = \sum_{n=-N}^{N} \abs{a_n}^2.
    \]
    De aquí se sigue que la serie \(\sum_{n=-N}^{N}\abs{a_n}^2\) converge, y además \(\sum_{n=-\infty}^{\infty}\abs{a_n}^2 \leq M^2 \). Ahora 
    podemos justificar el paso al límite cuando \(N \to \infty\) en la igualdad 
    \(\sum_{n=-N}^{N} \abs{a_n}^2 = \frac{1}{2\pi}\int_{0}^{2\pi} \abs{S_N(t)}^2 dt\), pues \(S_N\) converge uniformemente y la desigualdad 
    \(\abs{S_N(t)}\leq M\), por lo tanto la identidad de Parseval queda establecida. Ahora, para la especialización a las series de potencias
    consideremos un \(r<R\) y \(z = re^{it}\) arbitrarios. Basta con demostrar que la serie \(\sum_{n=0}^{\infty}a_nr^ne^{int}\) converge 
    uniformemente en \(\R\). Esto se sigue por la prueba \(M\) de Weierstrass, y de que las series de potencias convergen absolutamente dentro 
    de su disco de convergencia, ya que
    \[
    \abs{a_nr^n e^{int}} \leq \abs{a_n}r^n, \quad \text{y}\quad \sum_{n=0}^{\infty} \abs{a_n}r^n < \infty.
    \]

\end{dem}
