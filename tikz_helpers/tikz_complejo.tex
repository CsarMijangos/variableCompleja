% tikz_complejo.tex
%\usepackage{tikz}
%\usepackage{amsmath}
\usetikzlibrary{calc}
\usetikzlibrary{positioning}

% Estilo base para el plano complejo
\tikzset{
  complejo/.style={
    thick,
    opacity = 0.7,
    ->,
    >=stealth
  },
  punto/.style={
    circle,
    fill,
    inner sep=1pt
  },
  discoabierto/.style={
    draw=blue, thick, dashed, fill=blue!20, opacity=0.3
  },
  discocerrado/.style={
    draw=blue, thick, solid, fill=blue!20, opacity=0.4
  },
  regionConexa/.style={
    green!50, opacity=0.5, thick
  }
}

% Comando para dibujar el plano complejo
\newcommand{\planoComplejo}[1][]{
  \draw[complejo] (-3,0) -- (3,0) node[right] {\(x\)};
  \draw[complejo] (0,-3) -- (0,3) node[above] {\(y\)};
  % Opcional: título
  \ifx&#1&\else
    \node at (2.5,2.7) {#1};
  \fi
}

% Comando para dibujar el semi plano superior complejo
\newcommand{\semiPlanoComplejo}[1][]{
  \draw[complejo] (-3,0) -- (3,0) node[right] {\(x\)};
  \draw[complejo] (0,-0.5) -- (0,3) node[above] {\(y\)};
  % Opcional: título
  \ifx&#1&\else
    \node at (2.5,2.7) {#1};
  \fi
}

% Comando para dibujar el semi plano superior complejo con origen en punto dado
\newcommand{\semiPlanoComplejoCentrado}[2]{
  \draw[complejo] (#1 - 2.5,#2) -- (#1 + 2.5,#2) node[right] {\(x\)};
  \draw[complejo] (#1,#2 - 0.5) -- (#1,#2 + 2.5) node[above] {\(y\)};
 
}

% Disco abierto centrado en (x,y) con radio r
\newcommand{\discoAbierto}[4]{
  \filldraw[discoabierto] (#1,#2) circle (#3);
  \node at (#1+#4,#2+#4) {\(D(a,r)\)};
}

% Disco cerrado centrado en (x,y) con radio r
\newcommand{\discoCerrado}[4]{
  \filldraw[discocerrado] (#1,#2) circle (#3);
  \node at (#1+#4,#2+#4) {\(\overline{D}(a,r)\)};
}

% Punto etiquetado
\newcommand{\punto}[4]{
  \filldraw[punto] (#1,#2) circle (1pt) node[#3] {#4};
}

% Dibuja un segmento entre dos puntos, con etiqueta opcional
\newcommand{\segmentoRojo}[5]{
  \draw[thick, red] (#1,#2) -- (#3,#4);
  \ifx&#5&\else
    \node at ({(#1+#3)/2},{(#2+#4)/2 + 0.2}) {\(#5\)};
  \fi
}
\newcommand{\segmentoAzul}[5]{
  \draw[thick, blue] (#1,#2) -- (#3,#4);
  \ifx&#5&\else
    \node at ({(#1+#3)/2},{(#2+#4)/2 + 0.2}) {\(#5\)};
  \fi
}
\newcommand{\segmentoNegro}[5]{
  \draw[thick] (#1,#2) -- (#3,#4);
  \ifx&#5&\else
    \node at ({(#1+#3)/2},{(#2+#4)/2 + 0.2}) {\(#5\)};
  \fi
}

% Dibuja una región conexa con forma arbitraria a partir de coordenadas
% Uso: \regionConexa{{(x1,y1)(x2,y2)...}}{Etiqueta}
\newcommand{\regionConexa}[1]{
  \filldraw[regionConexa] plot [smooth cycle] coordinates #1;  
}
