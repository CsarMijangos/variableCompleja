% Cambiar fuentes para teoremas y demostraciones
\usepackage{lmodern}
\usepackage{titlesec}

% Estilo para teoremas
\newtheoremstyle{mytheorem}
  {10pt} % Espacio arriba
  {10pt} % Espacio abajo
  {\itshape} % Fuente del cuerpo
  {} % Sangría
  {\bfseries\color{blue}} % Fuente del título
  {} % Puntuación tras el título
  { } % Espacio tras el título
  {%
    \thmname{#1}~\thmnumber{#2}%
    \if\relax\detokenize{#3}\relax % Comprueba si la nota está vacía
    .\else~(\thmnote{#3}).\fi
  } % Formato del título

% Estilos numerados
\theoremstyle{mytheorem}
\newtheorem{theo}{Teorema}[chapter]
\newtheorem{lemma}[theo]{Lema}
\newtheorem{corol}[theo]{Corolario}
\newtheorem{conje}[theo]{Conjetura}
\newtheorem{ejem}{Ejemplo}[chapter]
\newtheorem{ejer}{Ejercicio}[chapter]

% Estilo para definiciones
\newtheoremstyle{mydefinition}
  {10pt}
  {10pt}
  {\normalfont} % Fuente normal (sin cursiva)
  {}
  {\bfseries\color{teal}}
  {}
  { }
  {%
    \thmname{#1}~\thmnumber{#2}%
    \if\relax\detokenize{#3}\relax % Comprueba si la nota está vacía
    \else~(\thmnote{#3}).\fi
  } % Formato del título

\theoremstyle{mydefinition}
\newtheorem*{defi}{Definición.}

% Estilo para demostraciones
\newtheoremstyle{mydemostration}
  {10pt}
  {10pt}
  {\normalfont} % Fuente normal (sin cursiva)
  {}
  {\bfseries}
  {}
  { }
  {\thmname{#1}~\thmnumber{#2}~\thmnote{(#3)}}

\theoremstyle{mydemostration}
\newtheorem*{dem}{Demostración.}
\renewcommand{\qedsymbol}{$\blacksquare$}
\AtEndEnvironment{dem}{\null\hfill\qedsymbol}
\newtheorem*{pd}{\textbf{P.D.}}

% Estilos no numerados
\theoremstyle{remark}
\newtheorem*{obs}{\textbf{Observación}}
\newtheorem*{notacion}{\textbf{Notación}}
\newtheorem*{afir}{\textbf{Afirmación}}



% Definir comandos personalizados
\newcommand{\R}{\mathbb{R}} % Conjunto de los números reales
\newcommand{\Z}{\mathbb{Z}} % Conjunto de los números enteros
\newcommand{\N}{\mathbb{N}} % Conjunto de los números naturales
\newcommand{\Q}{\mathbb{Q}} % Conjunto de los números racionales
\newcommand{\C}{\mathbb{C}} % Conjunto de los números complejos
\newcommand{\T}{\mathbb{T}} % El toro en C 
\newcommand{\Om}{\Omega} % Omega
\newcommand{\abs}[1]{\left\lvert#1\right\rvert} % Valor absoluto
\newcommand{\norm}[1]{\left\lVert#1\right\rVert} % Norma
\newcommand{\set}[1]{\left\{#1\right\}} % Conjunto
\newcommand{\inner}[2]{\langle#1, #2\rangle} % Producto interno

\newcommand{\derDefi}{\vcentcolon =} % Definición derecha :=
\newcommand{\izqDefi}{= \vcentcolon } % Definición inversa =:
\newcommand{\tq}{\, : \,} % tal que en conjuntos
\newcommand{\Real}{\mathtt{Re}} % Parte real de un complejo
\newcommand{\Imag}{\mathtt{Im}} % Parte imaginaria de un complejo
\newcommand*\conj[1]{\overline{#1}}  % Conjugado de un complejo


\newcommand*\closureSet[1]{\overline{#1}}  % Cerradura de un conjunto
%%% Acento para interior de conjuntos:
\DeclareFontFamily{U}{mathb}{\hyphenchar\font45}
\DeclareFontShape{U}{mathb}{m}{n}{ <-6> matha5 <6-7> matha6 <7-8>
mathb7 <8-9> mathb8 <9-10> mathb9 <10-12> mathb10 <12-> mathb12 }{}
\DeclareSymbolFont{mathb}{U}{mathb}{m}{n}

\DeclareMathAccent{\interiorSet}{0}{mathb}{"38}

\DeclareFontFamily{U}{mathb}{\hyphenchar\font45}
\DeclareFontShape{U}{mathb}{m}{n}{ <-6> matha5 <6-7> matha6 <7-8>
mathb7 <8-9> mathb8 <9-10> mathb9 <10-12> mathb10 <12-> mathb12 }{}
\DeclareSymbolFont{mathb}{U}{mathb}{m}{n}

%% Fin de acento para interior